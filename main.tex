\documentclass{article}
\usepackage{amsmath}
\usepackage{graphicx}
\usepackage{hyperref}
\title{Heat Exchanger Dashboard Documentation}
\author{}
\date{}
\begin{document}
\maketitle
\tableofcontents
\section{Introduction}
The dashboard is built with \texttt{streamlit} and provides an interactive environment to design and analyse shell--and--tube heat exchangers. It exposes two main components:
\begin{itemize}
  \item A \textbf{machine learning} interface predicting fouling level and the expected time to cleaning.
  \item A \textbf{design suite} for calculating geometry, shell clearance and thermal effectiveness using multiple methods.
\end{itemize}
\section{Machine Learning Predictor}
The page \texttt{pages/ml.py} defines the interface for the fouling model. Users input the operating conditions and obtain predictions for fouling and time--to--cleaning. Let $\mathbf{x} \in \mathbb{R}^{11}$ denote the feature vector
\begin{equation}
\mathbf{x}=\begin{bmatrix}
  t & \Delta T_h & \Delta T_c & \Delta P_{\text{shell}} & T_{h,\text{in}} & T_{c,\text{in}} \\ q_h & q_c & \mu_h & \mu_c & C_{s}
\end{bmatrix},
\end{equation}
where $t$ is the runtime since cleaning, $\Delta T$ are the measured temperature differences, $q$ are mass flow rates, $\mu$ are viscosities and $C_s$ is the solids concentration. The models loaded from \texttt{model\_fouling.pkl} and \texttt{model\_ttc.pkl} are applied as
\begin{equation}
F = f(\mathbf{x}), \qquad \tau = \max\bigl( g(\mathbf{x}),0 \bigr).
\end{equation}
Here $F$ denotes the predicted fouling level and $\tau$ the remaining hours before cleaning. The interface renders these values using \texttt{st.metric} widgets.
\section{Design Suite}
The main application in \texttt{app.py} offers tools for sizing the tube bundle, assessing shell clearance and computing thermal effectiveness. Important functions provided by the external package \texttt{ht.hx} include:
\begin{align}
D_{\text{bundle}} &= \text{size\_bundle\_from\_tubecount}(N, D_o, p, N_{tp}, \alpha),\\
N_{\text{tubes}} &= \text{Ntubes}(D_{\text{bundle}}, D_o, p, N_{tp}, \alpha),\\
\epsilon &= \text{temperature\_effectiveness\_\textit{type}}(R_1,\,\text{NTU},\,N_{tp},\,\text{option}),
\end{align}
where $N$ is the number of tubes, $D_o$ their outer diameter, $p$ the triangular pitch and $\alpha$ the arrangement angle. The thermal effectiveness functions cover TEMA E/G/H/J configurations, air coolers and plate exchangers.
\subsection{P--NTU Solver}
A general P--NTU method solves for outlet temperatures $T_{1,o}$ and $T_{2,o}$ using the heat balance
\begin{equation}
q = U A \Delta T_{\text{lm}} = \dot m_1 c_{p,1}(T_{1,i} - T_{1,o}) = \dot m_2 c_{p,2}(T_{2,o} - T_{2,i}).
\end{equation}
Given mass flow rates $\dot m_1$, $\dot m_2$, heat capacities $c_{p,1}$, $c_{p,2}$, inlet temperatures $T_{1,i}$, $T_{2,i}$ and the overall conductance $UA$, the function \texttt{P\_NTU\_method} computes the outlet temperatures consistent with a specific exchanger subtype.
\section{Usage}
Upon launching the dashboard with
\begin{verbatim}
streamlit run app.py
\end{verbatim}
users can expand each section to input parameters, trigger the computations and view the resulting metrics, tables and plots. The application keeps results in \texttt{st.session\_state} to allow comparison across runs.
\end{document}
